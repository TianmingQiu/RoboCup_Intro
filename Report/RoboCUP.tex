\documentclass[ICS, PP, english, final]{ICS_thesis}
\graphicspath{{pics/}{logos/}}


%_________MACROS_________ (optional and customizable - see output)
\input{./include/packages.tex}
% using the todonotes package in some nice ways (see packages.tex)
\newcommand{\add}[2][]{\todo[color=blue!40,#1]{#2}{}}
\newcommand\optional[2][]{\todo[inline, color=cyan!40, caption={2do},
	#1]{\begin{minipage}{\textwidth-4pt}#2\end{minipage}}{}}

% math macros
\newcommand{\set}[1]{\boldsymbol{#1}}
\renewcommand{\vec}[1]{\mathbf{#1}}

% % commands for easy referencing
\newcommand{\fref}[1]{Figure~\ref{#1}}
\newcommand{\tref}[1]{Table~\ref{#1}}
\newcommand{\eref}[1]{Equation~\ref{#1}}
\newcommand{\cref}[1]{Chapter~\ref{#1}}
\newcommand{\sref}[1]{Section~\ref{#1}}
\newcommand{\aref}[1]{Appendix~\ref{#1}}

% fancy source code listings: http://stackoverflow.com/questions/741985/latex-source-code-listing-like-in-professional-books
% usage: \lstinputlisting[label=samplecode,caption=A sample]{sourceCode/HelloWorld.java}
\definecolor{light-gray}{gray}{0.95}
\usepackage{listings}
\usepackage{courier}
\lstset{
         basicstyle=\footnotesize\ttfamily, % Standardschrift
         numbers=left,               % Ort der Zeilennummern
         numberstyle=\tiny,          % Stil der Zeilennummern
         stepnumber=0,               % Abstand zwischen den Zeilennummern
         numbersep=5pt,              % Abstand der Nummern zum Text
         tabsize=2,                  % Groesse von Tabs
         extendedchars=true,         %
         breaklines=true,            % Zeilen werden Umgebrochen
         keywordstyle=\color{red},
         stringstyle=\color{white}\ttfamily, % Farbe der String
         showspaces=false,           % Leerzeichen anzeigen ?
         showtabs=false,             % Tabs anzeigen ?
         xleftmargin=0pt,
         framexleftmargin=10pt,
         framexrightmargin=10pt,
         framexbottommargin=0pt,
         backgroundcolor=\color{light-gray},
         showstringspaces=false      % Leerzeichen in Strings anzeigen ?
}

\lstdefinestyle{customc}{
  	belowcaptionskip=1\baselineskip,
  	breaklines=true,
  	%frame=L,
  	language=C,
  	showstringspaces=false,
 	basicstyle=\footnotesize\ttfamily\color{blue!40!black},
 	keywordstyle=\bfseries\color{green!40!black},
  	commentstyle=\itshape\color{purple!40!black},
  	%identifierstyle=\color{blue}, %color of actual code
 	stringstyle=\color{orange},
    numbers=left,               % Ort der Zeilennummern
    numberstyle=\tiny,          % Stil der Zeilennummern
    stepnumber=2,               % Abstand zwischen den Zeilennummern
    numbersep=5pt,              % Abstand der Nummern zum Text
    tabsize=2,                  % Groesse von Tabs
    extendedchars=true,         %
    showspaces=false,           % Leerzeichen anzeigen ?
    showtabs=false,             % Tabs anzeigen ?
  	xleftmargin=\parindent,
    framexleftmargin=10pt,
    framexrightmargin=10pt,
    framexbottommargin=0pt,
    backgroundcolor=\color{light-gray},
}

\lstset{escapechar=@,style=customc}

%_______Start_Document______________________________________
\begin{document}


\title{RoboCup SS17}

\student{B.Sc. Tianming Qiu} 			%% your name
\yearofbirth{30.03.1993}	                	%% date of birth
\street{Felsennelkenanger 15}			%% your address
\city{80937, Munich}						%		"
\phone{+49 17643387325}					%% your telephone-no.
\supervisor{Mohsen Kaboli}				%% your supervisor
\start{24.04.2017}						%% start date
\finalrep{24.07.2017}					%% final presentation / date

\maketitle



% ________Abstract__________________________________________-
\topmargin5mm
\textheight220mm
\pagenumbering{arabic}
\phantom{u}
\begin{abstract}
RoboCup is a competition for soccer robot held annually, which aims at promoting researches on robot and artificial intelligence. We work on the RoboCup Standard Platform League (SPL) by using humanoid robot NAO, based on the code framework from University of Bremen's team BHuman.

Self-localization is a very important task for NAO during the game. The action decisions, ball tracking and different team cooperation strategies are dependent on the precise location on the standard soccer field. Self-localization is also related computer vision and statistical signal processing. These topics are the foundations of autonomous robot. So this semester I focus on figuring it out, how does NAO localize itself on the soccer field by cameras and how could it be improved.

The current method in BHuman framework has been continuously developed, verified and implemented for several years and it has been considered as state-of-the-art method which is already very robust and compact. The main idea is to apply particle filter to estimate robot pose. But there are still some issues that when robot is walking, some shake noises from camera will cause deviations in visual odometry.
\end{abstract}


% %%%%%%%%%%%%%%%%%%%%% Widmung %%%%%%%%%%%%%%%%%%%%%%%%%%%%%%%%
\phantom{u}
\phantom{1}\vspace{6cm}
\begin{center}
%Hier die Widmung oder leer lassen
\end{center}


\pagestyle{fancy}

%%%%%%%%%%%%%%%%%%%Inhaltsverzeichnis%%%%%%%%%%%%%%%%%%%%%%%%%%
\tableofcontents

%%%%%%%%%%%%%%%%%%%%%%%%%%%%%%%
% ACTUAL CONTENT OF YOUR WORK %
%%%%%%%%%%%%%%%%%%%%%%%%%%%%%%%
%%%%%%%%%%%% Kapitel - externe Dateien zur Ordnung%%%%%%%%%%%%%
\chapter{Introduction}
RoboCup is a competition for the soccer robot held annually, which aims at promoting researches on robots and artificial intelligence. We work on the RoboCup Standard Platform League (SPL) by using humanoid robot NAO. There are 5 players for each side: goalkeeper, defender, supporter and striker. Different roles share some same functions such as self-localization, tracking the ball, kicking and so on. They also have different strategies according to their distributed job on the field just like the real soccer match. Of course the necessary communication and cooperation are required to achieve better performance in the match. One of the most successful team is B-Human from University of Bremen, who have won the championship five times. Last year TU Munich also set up our own RoboCup team called ``TUM Lion".
\section{Motivation}
The course ``Introduction Lab Humanoid RoboCup" in summer semester 2017 is given for the students are interested in RoboCup and gives them a brief idea about RoboCup as well as the general knowledge of practice on autonomous robot localization, computer vision, motion control and multi-robot cooperation tasks. This semester's project is based on the code framework from University of Bremen's team B-Human. With the help of the open source framework could let us start the project more quickly and do some specific research on the topic what we are interested in or more familiar with. At the same time, some issues will be found and could be improved or considered as another new strategy.

\section{My Work}
Since we have already got the access to the state-of-the-art open source framework, so the first stage is to get familiar with the procedure of NAO operating and the codes. It includes:
\begin{itemize}
   	\item Basic operations on NAO and remote control through SimRobot
   	\item Calibration(including camera parameters, color and initial motions)
	\item Understand how the current framework realize the self-localization
\end{itemize}

The second stage is focusing on self-localization and finding some interesting part to improve. Firstly I learned the Markov localization theory of mobile robot such as particle filter and unscented Kalman filter, which is used in the B-Human framework. Then the PnP pose estimation method is adopted to eliminate the current visual odometry error. In Chapter 2 I will introduce how does the current method figure out the localization of robot and the procedure of particle filer and unscented Kalman filter. In Chapter 3 I will show the visual odometry procedure of B-Human and explain what kind of noise it could be as well as the method that I applied.


\chapter{Improvement of Visual Odometry}\label{Chap:Imp}
Above the procedure of self-localization was described in general. Particle filter require sensor measurement to calculate the weighting for each particle and Unscented Kalman filter demands the sensor measurements to update the motion predictions. Visual odometry is a widely used with strong functions sensor measurement method, which is applied in this situation and provides the measurement for each kind of filter. So the accuracy of pose estimated by visual part will definitely have an obvious influence on the final result. This chapter will introduce what knowledge of the sensor measurement can provide and how the sensor measurement take place. 
\section{The current method}
\subsection{Visual odometry procedure}
There are several sensors to acquire information from the environment on NAO: camera, microphone and sonar. From them only the camera is adopted, since the environment of the soccer match will be quite complicated, for example:
\begin{itemize}
    \item The white border lines and green field are on the same plane.
    \item The goal posts are perhaps too far away from the robots sometimes.
    \item There are 6 moving robots on the field, which the opponents are distinguished from our teammates only by colors.
    \item The ball is small and the detection of a ball requires high precision.
\end{itemize}
So based on above reasons, the another sensors are not qualified for the requirements. However, the camera is suitable for each specific task and the computer vision skill as well as visual odometry are being perfect so far.

There two cameras on the NAO's head which can provide large FOV. Several tasks such as image preprocessing, feature detection(including: line perception, penalty mark perception, ball perception) and data association will be performed on the image gathered by each camera. B-Human's current vision system provides a variety of perceptions that have all been integrated into the sensor model: goal posts (ambiguous as well as unambiguous ones), line segments (of which only the endpoints are matched to the field model), line crossings (of three different types: L, T, and X ), and the center circle \cite{BHumanCodeRelease2012}. These features can be associated to the feature detected in pixel images. The detailed of these tasks will not be mentioned in my part of report, which are introduced by my teammates. My work is that suppose the the feature detection and data association have been finished.

Based on feature detection and data association, the corresponding point pairs have been already found. For example the goal post in figure i. whose coordinate in the homogeneous global 3D world could be represented as $(X,Y,0,1)^T$. And the corresponding homogeneous coordinate in pixel coordinate can be written as $(x,y,1)^T$. Since the camera calibration matrix, including both intrinsic and extrinsic calibration parameters, has been acquired through camera calibration before each match. Assuming the camera model as pinhole camera model shown in \fref{fig: camera} \cite{hartley2003multiple}: 
\[ % without number
\mathbf{x} = \mathbf{K} \cdot \mathbf{X}_{cam}
\]
\begin{figure}[!htb]
    \includegraphics[width=0.4\textwidth]{pics/cameramodel.png}
    \centering
    \caption{Pinhole camera model}
    \label{fig: camera}
\end{figure}

So we could get the corresponding 3D goal post coordinate in the ideal camera frame. Since we know the geometrical value of NAO's body construction, we can transform the coordinates into the robot frame, which is shown in the \fref{fig: trans.sub.1}
\begin{figure}[tbp]
\centering
\subfigure[Coordinate transform]{
\label{trans.sub.1}
\includegraphics[width=0.45\textwidth]{pics/coordinate_transform.png}}
\subfigure[Relative pose to landmark]{
\label{trans.sub.2}
\includegraphics[width=0.45\textwidth]{pics/relative.png}}
\caption{Pose estimation}
\label{fig: trans}
\end{figure}
Based on this, we could get the related distance between robot and the goal post in robot frame. At the same time, the precise location of goal post in the global field is also known. So we could calculate the robot pose from this above information.
\subsection{Current problem}
In the process of transforming the coordinate from camera frame into robot frame, the current method only use a constant transform. However when robot is walking, it cannot always be extremely vertical to the ground. The central line of its body and camera on the head will have an arbitrary angle of tilt. Also due to detecting of the environment, robot always turn around its head or even look up and down, but this information will also create bias by coordinate transforming. Then it will cause a bias on the camera frame transform, which is shown in \fref{fig: error}. This error between the real robot base coordinate and the result calculated by the constant transform will further affect the pose estimation. 
\begin{figure}[!htb]
    \includegraphics[width=0.7\textwidth]{pics/error.png}
    \centering
    \caption{Position error by a constant transform}
    \label{fig: error}
\end{figure}\\

\section{PnP Pose estimation}
Since the current method contains the tilting and shaking error I apply the PnP method \cite{ETHPnP} which could eliminate the error by considering the transform in a more direct way. PnP means ``Point n Point", which uses $n$ pairs of corresponding point to calculate transform relationship. Reconsidering the corresponding point pair $\mathbf{x}:=(x,y,1)^T$ in pixel coordinate and $\mathbf{X}:=(X,Y,0,1)^T$ in 3D global space. They should have this following relationship \eqref{relation}:
\begin{equation}
\mathbf{x} = \mathbf{T}_{image2field} \cdot \mathbf{X}_{field} \label{relation}
\end{equation}
And the transform matrix $\mathbf{T}_{image2field} \in \mathfrak{R}^{3 \times 4}$ could be represented as three parts: 
\begin{equation}
\mathbf{T}_{image2field}=\mathbf{K} \cdot \mathbf{T}_{camera2robot} \cdot \mathbf{T}_{robot2field} \label{transform}
\end{equation}
In \eqref{transform} what we can figure out from the corresponding point pairs is $\mathbf{T}_{image2field}$, and what we want to know is $\mathbf{T}_{robot2field}$. If we know the $\mathbf{T}_{robot2field}$ then we could directly read robot pose from this matrix. Also, the camera calibration matrix $\mathbf{K}$ is also already known because the calibration has been done before each match. 
$\mathbf{T}_{robot2field}$ can also be represented as \eqref{Trf}:
\begin{equation}\mathbf{T}_{robot2field}=
\begin{bmatrix}
\mathbf{R} & \mathbf{T}
\end{bmatrix}=
\begin{bmatrix}
cos\alpha & sin\alpha & 0 & T_x\\
-sin\alpha & cos\alpha & 0 & T_y\\
0 & 0 & 1 & 0
\end{bmatrix} \label{Trf}
\end{equation}

There are 4 unknown parameters in \eqref{Trf}, so at least 2 pairs of corresponding points are required to solve the equation. The 2 pairs of corresponding points can be selected by two goal posts or the start and end point of a specific line. Suppose $\mathbf{M}:=\mathbf{K} \cdot \mathbf{T}_{camera2robot}$, elements in $\mathbf{M}$ are all known values, so the equation can be written as \eqref{Trf2}:
\begin{equation}\mathbf{x}=\mathbf{M} \cdot
\begin{bmatrix}
\mathbf{R} & \mathbf{T}
\end{bmatrix} \cdot \mathbf{X} =\mathbf{A}\cdot \mathbf{X}
\label{Trf2}
\end{equation}

\begin{equation}[\mathbf{x_1} \quad \mathbf{x_2}]=
\mathbf{A}\cdot [\mathbf{X_2} \quad \mathbf{X_2}]
\label{Trf3}
\end{equation}

So, in \eqref{Trf3} it shows the how did two pairs of corresponding points calculate matrix $\mathbf{A}$ which contains four unknown parameters. If we have more than 4 pairs of corresponding points, SVD method for minimum solution finding could be applied.







% \appendix
% 	\input{./chapters/Appendix.tex}

%%%%%%%%%%%%%%%%%%Acknoledgments %%%%%%%%%%%%%%%%%%%%%%
\cleardoublepage
\chapter*{Acknowledgments}
\markright{ACKNOWLEDGMENTS}
I would like to express my thanks to Prof. Dr. Gordon Cheng and Chair of Cognitive Systems for offering this course and all the hardware support.

I would further like to thank our supervisor Mohsen Kaboli for your patient help and many constructive suggestions. You are always willing to share your experiences and give us significant guide. Your office is always open for us.
I would also appreciate the help offered by our teaching assistant Zhiliang Wu, who prepared all the lab stuff fur us and wrote detailed tutorial that let me know how to operate NAO step by step. Your careful work make me get start quickly and give me a lot confidence to finish the task. 

I am particularly thankful to my team mate: Yao, Fabian, Zhiyi, Jingjie, and Minkai. We discussed a lot and you always give me some useful supports.

%%%%%%%%%%%%%%%%%%_Abbildungsverzeichnis %%%%%%%%%%%%%%%%%%%%%%
\cleardoublepage
\addcontentsline{toc}{chapter}{List of Figures}
\listoffigures

% %%%%%%%%%%%%%%%%%%_Acronyms and Notations %%%%%%%%%%%%%%%%%%%%%%
% \cleardoublepage
% \chapter*{Acronyms and Notations}
% \input{./include/acronyms.tex}

%%%%%%%%%%%%%%%%%%Literaturverzeichnis %%%%%%%%%%%%%%%%%%%%%%%%
\cleardoublepage
\addcontentsline{toc}{chapter}{Bibliography}
\bibliography{mybib}
\bibliographystyle{unsrt}

%%%%%%%%%%%%%%%%%%%%License %%%%%%%%%%%%%%%%%%%%%%%%%%%%%%%%%%%
\cleardoublepage
\chapter*{License}
\markright{LICENSE}
This work is licensed under the Creative Commons Attribution 3.0 Germany
License. To view a copy of this license,
visit \href{http://creativecommons.org/licenses/by/3.0/de/}{http://creativecommons.org} or send a letter
to Creative Commons, 171 Second Street, Suite 300, San
Francisco, California 94105, USA.

\end{document}
