\documentclass[ICS, PP, english, final]{ICS_thesis}
\graphicspath{{pics/}{logos/}}


%_________MACROS_________ (optional and customizable - see output)
\input{./include/packages.tex}
% using the todonotes package in some nice ways (see packages.tex)
\newcommand{\add}[2][]{\todo[color=blue!40,#1]{#2}{}}
\newcommand\optional[2][]{\todo[inline, color=cyan!40, caption={2do},
	#1]{\begin{minipage}{\textwidth-4pt}#2\end{minipage}}{}}

% math macros
\newcommand{\set}[1]{\boldsymbol{#1}}
\renewcommand{\vec}[1]{\mathbf{#1}}

% % commands for easy referencing
\newcommand{\fref}[1]{Figure~\ref{#1}}
\newcommand{\tref}[1]{Table~\ref{#1}}
\newcommand{\eref}[1]{Equation~\ref{#1}}
\newcommand{\cref}[1]{Chapter~\ref{#1}}
\newcommand{\sref}[1]{Section~\ref{#1}}
\newcommand{\aref}[1]{Appendix~\ref{#1}}

% fancy source code listings: http://stackoverflow.com/questions/741985/latex-source-code-listing-like-in-professional-books
% usage: \lstinputlisting[label=samplecode,caption=A sample]{sourceCode/HelloWorld.java}
\definecolor{light-gray}{gray}{0.95}
\usepackage{listings}
\usepackage{courier}
\lstset{
         basicstyle=\footnotesize\ttfamily, % Standardschrift
         numbers=left,               % Ort der Zeilennummern
         numberstyle=\tiny,          % Stil der Zeilennummern
         stepnumber=0,               % Abstand zwischen den Zeilennummern
         numbersep=5pt,              % Abstand der Nummern zum Text
         tabsize=2,                  % Groesse von Tabs
         extendedchars=true,         %
         breaklines=true,            % Zeilen werden Umgebrochen
         keywordstyle=\color{red},
         stringstyle=\color{white}\ttfamily, % Farbe der String
         showspaces=false,           % Leerzeichen anzeigen ?
         showtabs=false,             % Tabs anzeigen ?
         xleftmargin=0pt,
         framexleftmargin=10pt,
         framexrightmargin=10pt,
         framexbottommargin=0pt,
         backgroundcolor=\color{light-gray},
         showstringspaces=false      % Leerzeichen in Strings anzeigen ?
}

\lstdefinestyle{customc}{
  	belowcaptionskip=1\baselineskip,
  	breaklines=true,
  	%frame=L,
  	language=C,
  	showstringspaces=false,
 	basicstyle=\footnotesize\ttfamily\color{blue!40!black},
 	keywordstyle=\bfseries\color{green!40!black},
  	commentstyle=\itshape\color{purple!40!black},
  	%identifierstyle=\color{blue}, %color of actual code
 	stringstyle=\color{orange},
    numbers=left,               % Ort der Zeilennummern
    numberstyle=\tiny,          % Stil der Zeilennummern
    stepnumber=2,               % Abstand zwischen den Zeilennummern
    numbersep=5pt,              % Abstand der Nummern zum Text
    tabsize=2,                  % Groesse von Tabs
    extendedchars=true,         %
    showspaces=false,           % Leerzeichen anzeigen ?
    showtabs=false,             % Tabs anzeigen ?
  	xleftmargin=\parindent,
    framexleftmargin=10pt,
    framexrightmargin=10pt,
    framexbottommargin=0pt,
    backgroundcolor=\color{light-gray},
}

\lstset{escapechar=@,style=customc}

%_______Start_Document______________________________________
\begin{document}


\title{RoboCup SS17}

\student{B.Sc. Tianming Qiu} 			%% your name
\yearofbirth{30.03.1993}	                	%% date of birth
\street{Felsennelkenanger 15}			%% your address
\city{80937, Munich}						%		"
\phone{+49 17643387325}					%% your telephone-no.
\supervisor{Mohsen Kaboli}				%% your supervisor
\start{24.04.2017}						%% start date
\finalrep{24.07.2017}					   	%% final presentation / date

\maketitle



% ________Abstract__________________________________________-
\topmargin5mm
\textheight220mm
\pagenumbering{arabic}
\phantom{u}
\begin{abstract}
RoboCup is a competition for soccer robot held annually and we work on the RoboCup Standard Platform League (SPL) by using humanoid robot NAO, based on the code framework from University of Bremen's team B-Human.

Self-localization is a very important task for NAOs during the match. The action decisions, ball tracking and different team cooperation strategies are dependent on the precise location on the standard soccer field. Self-localization is also related computer vision and statistical signal processing. These topics are the foundations of autonomous robot. So this semester I focus on figuring it out, how does NAO localize itself on the soccer field by cameras and how could it be improved.

The current method in B-Human framework has been continuously developed, verified and implemented for several years. It has been considered as state-of-the-art method which is already very robust and compact. The main idea is to apply particle filter with unscented Kalman filter for each particle to estimate robot pose. But there are still some issues that when robot is walking, some shake noises from camera will cause deviations in visual odometry. PnP pose estimate would be adopted instead of relative distance calculation, which would take all the transform generally into consideration and eliminates the error.
\end{abstract}

% %%%%%%%%%%%%%%%%%%%%% Widmung %%%%%%%%%%%%%%%%%%%%%%%%%%%%%%%%
\phantom{u}
\phantom{1}\vspace{6cm}
\begin{center}
%Hier die Widmung oder leer lassen
\end{center}


\pagestyle{fancy}

%%%%%%%%%%%%%%%%%%%Inhaltsverzeichnis%%%%%%%%%%%%%%%%%%%%%%%%%%
\tableofcontents

%%%%%%%%%%%%%%%%%%%%%%%%%%%%%%%
% ACTUAL CONTENT OF YOUR WORK %
%%%%%%%%%%%%%%%%%%%%%%%%%%%%%%%
%%%%%%%%%%%% Kapitel - externe Dateien zur Ordnung%%%%%%%%%%%%%
\chapter{Introduction}
RoboCup is a competition for the soccer robot held annually, which aims at promoting researches on robots and artificial intelligence. We work on the RoboCup Standard Platform League (SPL) by using humanoid robot NAO. There are 5 players for each side: goalkeeper, defender, supporter and striker. Different roles share some same functions such as self-localization, tracking the ball, kicking and so on. They also have different strategies according to their distributed job on the field just like the real soccer match. Of course the necessary communication and cooperation are required to achieve better performance in the match. One of the most successful team is B-Human from University of Bremen, who have won the championship five times. Last year TU Munich also set up our own RoboCup team called ``TUM Lion".
\section{Motivation}
The course ``Introduction Lab Humanoid RoboCup" in summer semester 2017 is given for the students are interested in RoboCup and gives them a brief idea about RoboCup as well as the general knowledge of practice on autonomous robot localization, computer vision, motion control and multi-robot cooperation tasks. This semester's project is based on the code framework from University of Bremen's team B-Human. With the help of the open source framework could let us start the project more quickly and do some specific research on the topic what we are interested in or more familiar with. At the same time, some issues will be found and could be improved or considered as another new strategy.

\section{My Work}
Since we have already got the access to the state-of-the-art open source framework, so the first stage is to get familiar with the procedure of NAO operating and the codes. It includes:
\begin{itemize}
   	\item Basic operations on NAO and remote control through SimRobot
   	\item Calibration(including camera parameters, color and initial motions)
	\item Understand how the current framework realize the self-localization
\end{itemize}

The second stage is focusing on self-localization and finding some interesting part to improve. Firstly I learned the Markov localization theory of mobile robot such as particle filter and unscented Kalman filter, which is used in the B-Human framework. Then the PnP pose estimation method is adopted to eliminate the current visual odometry error. In Chapter 2 I will introduce how does the current method figure out the localization of robot and the procedure of particle filer and unscented Kalman filter. In Chapter 3 I will show the visual odometry procedure of B-Human and explain what kind of noise it could be as well as the method that I applied.

\chapter{Self-Localization}\label{Chap:PF}
This chapter introduces the concrete method used in the current framework for NAO, the mobile roboter lacalization. RoboCup self-localization is a problem that NAO robot could determine its pose, which includes both position and angular, relative to the given map of the standard soccer field. Almost all the tasks for each robot player require knowledge of its pose on the field. We call this problem as rather a self-localization than a SLAM(Simultaneous localization and mapping) problem, one important reason is the map of the environment has been provided as prior knowledges. Self-localization could be treated as a problem of coordinate transformation. Map of the field or environment is described as the global coordinate system, which does not rely on robot's pose. The robot pose will be finally transformed into the global frame and represented as a global coordiante\cite{thrun2005probabilistic}. In our project the NAO pose is only changes in a plane, the $z$ element is always remains as zero since the robot always stands firmly on the ground. So the pose could be written as $(x,y,\theta)^T$.

However, the pose of robot could not be aquired directly, instead it could be inferred from the sensor data, which here come from camera. The sensor could not be hundred percentage accurate, it alway contains noise, especially image processing by using camera. So we have to find a proper method to calculate the pose from imperfect sensor data, which will be introduced in this chapter.

\clearpage
\section{The current method}
\subsection{Visual odometry procedure}
ave been finished.


So we could get the corresponding goal post coordinate in the ideal camera frame. Since we know the geometrical value of NAO's body construction, we can transform the coordinates into the robot frame, which is shown in the figure % coordinate transform
Based on this, we could get the related distance between robot and the goal post in robot frame. At the same time, the pricise location of goal post in the global field is also known. So we could calculate the robot pose from this above information. % show the calculation procedure

\subsection{Current problem}
In the process of transforming the coordinate from camera frame into robot frame, the current method only use a constant transform. However

\chapter{Improvement of Visual Odometry}\label{Chap:Imp}
Above the procedure of self-localization was described in general. Particle filter require sensor measurement to calculate the weighting for each particle and Unscented Kalman filter demands the sensor measurements to update the motion predictions. Visual odometry is a widely used with strong functions sensor measurement method, which is applied in this situation and provides the measurement for each kind of filter. So the accuracy of pose estimated by visual part will definitely have an obvious influence on the final result. This chapter will introduce what knowledge of the sensor measurement can provide and how the sensor measurement take place. 
\section{The current method}
\subsection{Visual odometry procedure}
There are several sensors to acquire information from the environment on NAO: camera, microphone and sonar. From them only the camera is adopted, since the environment of the soccer match will be quite complicated, for example:
\begin{itemize}
    \item The white border lines and green field are on the same plane.
    \item The goal posts are perhaps too far away from the robots sometimes.
    \item There are 6 moving robots on the field, which the opponents are distinguished from our teammates only by colors.
    \item The ball is small and the detection of a ball requires high precision.
\end{itemize}
So based on above reasons, the another sensors are not qualified for the requirements. However, the camera is suitable for each specific task and the computer vision skill as well as visual odometry are being perfect so far.

There two cameras on the NAO's head which can provide large FOV. Several tasks such as image preprocessing, feature detection(including: line perception, penalty mark perception, ball perception) and data association will be performed on the image gathered by each camera. B-Human's current vision system provides a variety of perceptions that have all been integrated into the sensor model: goal posts (ambiguous as well as unambiguous ones), line segments (of which only the endpoints are matched to the field model), line crossings (of three different types: L, T, and X ), and the center circle \cite{BHumanCodeRelease2012}. These features can be associated to the feature detected in pixel images. The detailed of these tasks will not be mentioned in my part of report, which are introduced by my teammates. My work is that suppose the the feature detection and data association have been finished.

Based on feature detection and data association, the corresponding point pairs have been already found. For example the goal post in figure i. whose coordinate in the homogeneous global 3D world could be represented as $(X,Y,0,1)^T$. And the corresponding homogeneous coordinate in pixel coordinate can be written as $(x,y,1)^T$. Since the camera calibration matrix, including both intrinsic and extrinsic calibration parameters, has been acquired through camera calibration before each match. Assuming the camera model as pinhole camera model shown in \fref{fig: camera} \cite{hartley2003multiple}: 
\[ % without number
\mathbf{x} = \mathbf{K} \cdot \mathbf{X}_{cam}
\]
\begin{figure}[!htb]
    \includegraphics[width=0.4\textwidth]{pics/cameramodel.png}
    \centering
    \caption{Pinhole camera model}
    \label{fig: camera}
\end{figure}

So we could get the corresponding 3D goal post coordinate in the ideal camera frame. Since we know the geometrical value of NAO's body construction, we can transform the coordinates into the robot frame, which is shown in the \fref{fig: trans.sub.1}
\begin{figure}[tbp]
\centering
\subfigure[Coordinate transform]{
\label{trans.sub.1}
\includegraphics[width=0.45\textwidth]{pics/coordinate_transform.png}}
\subfigure[Relative pose to landmark]{
\label{trans.sub.2}
\includegraphics[width=0.45\textwidth]{pics/relative.png}}
\caption{Pose estimation}
\label{fig: trans}
\end{figure}
Based on this, we could get the related distance between robot and the goal post in robot frame. At the same time, the precise location of goal post in the global field is also known. So we could calculate the robot pose from this above information.
\subsection{Current problem}
In the process of transforming the coordinate from camera frame into robot frame, the current method only use a constant transform. However when robot is walking, it cannot always be extremely vertical to the ground. The central line of its body and camera on the head will have an arbitrary angle of tilt. Also due to detecting of the environment, robot always turn around its head or even look up and down, but this information will also create bias by coordinate transforming. Then it will cause a bias on the camera frame transform, which is shown in \fref{fig: error}. This error between the real robot base coordinate and the result calculated by the constant transform will further affect the pose estimation. 
\begin{figure}[!htb]
    \includegraphics[width=0.7\textwidth]{pics/error.png}
    \centering
    \caption{Position error by a constant transform}
    \label{fig: error}
\end{figure}\\

\section{PnP Pose estimation}
Since the current method contains the tilting and shaking error I apply the PnP method \cite{ETHPnP} which could eliminate the error by considering the transform in a more direct way. PnP means ``Point n Point", which uses $n$ pairs of corresponding point to calculate transform relationship. Reconsidering the corresponding point pair $\mathbf{x}:=(x,y,1)^T$ in pixel coordinate and $\mathbf{X}:=(X,Y,0,1)^T$ in 3D global space. They should have this following relationship \eqref{relation}:
\begin{equation}
\mathbf{x} = \mathbf{T}_{image2field} \cdot \mathbf{X}_{field} \label{relation}
\end{equation}
And the transform matrix $\mathbf{T}_{image2field} \in \mathfrak{R}^{3 \times 4}$ could be represented as three parts: 
\begin{equation}
\mathbf{T}_{image2field}=\mathbf{K} \cdot \mathbf{T}_{camera2robot} \cdot \mathbf{T}_{robot2field} \label{transform}
\end{equation}
In \eqref{transform} what we can figure out from the corresponding point pairs is $\mathbf{T}_{image2field}$, and what we want to know is $\mathbf{T}_{robot2field}$. If we know the $\mathbf{T}_{robot2field}$ then we could directly read robot pose from this matrix. Also, the camera calibration matrix $\mathbf{K}$ is also already known because the calibration has been done before each match. 
$\mathbf{T}_{robot2field}$ can also be represented as \eqref{Trf}:
\begin{equation}\mathbf{T}_{robot2field}=
\begin{bmatrix}
\mathbf{R} & \mathbf{T}
\end{bmatrix}=
\begin{bmatrix}
cos\alpha & sin\alpha & 0 & T_x\\
-sin\alpha & cos\alpha & 0 & T_y\\
0 & 0 & 1 & 0
\end{bmatrix} \label{Trf}
\end{equation}

There are 4 unknown parameters in \eqref{Trf}, so at least 2 pairs of corresponding points are required to solve the equation. The 2 pairs of corresponding points can be selected by two goal posts or the start and end point of a specific line. Suppose $\mathbf{M}:=\mathbf{K} \cdot \mathbf{T}_{camera2robot}$, elements in $\mathbf{M}$ are all known values, so the equation can be written as \eqref{Trf2}:
\begin{equation}\mathbf{x}=\mathbf{M} \cdot
\begin{bmatrix}
\mathbf{R} & \mathbf{T}
\end{bmatrix} \cdot \mathbf{X} =\mathbf{A}\cdot \mathbf{X}
\label{Trf2}
\end{equation}

\begin{equation}[\mathbf{x_1} \quad \mathbf{x_2}]=
\mathbf{A}\cdot [\mathbf{X_2} \quad \mathbf{X_2}]
\label{Trf3}
\end{equation}

So, in \eqref{Trf3} it shows the how did two pairs of corresponding points calculate matrix $\mathbf{A}$ which contains four unknown parameters. If we have more than 4 pairs of corresponding points, SVD method for minimum solution finding could be applied.






\chapter{Conclusion}
\section{Results}
In the preliminary game with the other team, we lose the game of 2:3, as our striker are not working properly. After a long time tuning and testing, we fixed the existing problems for calibration and localization and implemented our new behavior for the defender and striker. As a result, in the last game on the presentation day, our yellow team beat the other team with 2:0.
\section{Suggestion for the future work}
Although there are already some nice improvement this semester, more work need to be done in the future to get better performance. Here are two main ideas for the role behavior improvement for the future:
\begin{itemize}
    \item Dynamic role assignment
    \item ``Intelligent'' choice of kick direction
\end{itemize}
\subsection{Dynamic role assignment}
The situation of a real game changes really fast. To achieve better result, the dynamic role assignment needs to be implemented. More concretely, the robots will decide by himself which role he is playing. If the position of the defender is over the supporter, their roles could be exchanged immediately. So the original defender don't need to wasting time going back and the original supporter don't need to going forward again to keep their ``set'' behavior. In addition, if the supporter is ahead of striker by some reason, he should automatically work as the striker as in that situation he can score easier.
With this improvement, the team will work more efficiently and more dynamic strategies can be found.
\subsection{``Intelligent'' choice of kick direction}
The proposed method in Section~\ref{sec:Striker} is already a nice improvement over the original kick direction. But if the opponent goal-keeper react quickly and strongly enough, he may still be able to stop it. As shown in \fref{fig:IntKik}, the direction of the kicking should be determined by calculating the distance of the goalkeeper to different goal posts(marked with red dotted lines). If the goalkeeper is near to the left post like the shown graph, the kicking direction should be the right goal post. With this method the goalkeeper will not be able to stop the ball as he is already too far away from the other goal post.
\begin{figure}[!htb]
    \includegraphics[scale = 0.3]{pics/striker_future}
    \centering
    \caption{``Intelligent'' choice of kick direction}
    \label{fig:IntKik}
\end{figure}


% \appendix
% 	\input{./chapters/Appendix.tex}

%%%%%%%%%%%%%%%%%%Acknoledgments %%%%%%%%%%%%%%%%%%%%%%
%\cleardoublepage
\chapter*{Acknowledgments}
\markright{ACKNOWLEDGMENTS}
First of all, I would like to express my thanks to Prof. Dr. Gordon Cheng and Chair of Cognitive Systems for offering this course and all the hardware equipments.

I would further like to thank our supervisor Mohsen Kaboli for your patient help and many constructive suggestions. You are always willing to share your experiences and give us significant guide. Your office is always open to us. You are strict to us but I really learn a lot from that. Not only the experiment itself also the way how to deliver a presentation.

Also, I would appreciate the help offered by our teaching assistant Zhiliang Wu, who prepared all the lab stuff fur us and wrote detailed tutorial that let me know how to operate NAO step by step. Your careful work make me get start quickly and give me a lot confidence to finish the task. 

I am particularly thankful to my team mate: Yao, Fabian, Zhiyi, Jingjie, and Minkai. We discussed a lot and you always give me some useful supports.

%%%%%%%%%%%%%%%%%%_Abbildungsverzeichnis %%%%%%%%%%%%%%%%%%%%%%
\cleardoublepage
\addcontentsline{toc}{chapter}{List of Figures}
\listoffigures

% %%%%%%%%%%%%%%%%%%_Acronyms and Notations %%%%%%%%%%%%%%%%%%%%%%
 %\cleardoublepage
% \chapter*{Acronyms and Notations}
% \input{./include/acronyms.tex}

%%%%%%%%%%%%%%%%%%Literaturverzeichnis %%%%%%%%%%%%%%%%%%%%%%%%
\cleardoublepage
\addcontentsline{toc}{chapter}{Bibliography}
\bibliography{mybib}
\bibliographystyle{unsrt}

%%%%%%%%%%%%%%%%%%%%License %%%%%%%%%%%%%%%%%%%%%%%%%%%%%%%%%%%
%\cleardoublepage
%\chapter*{License}
%\markright{LICENSE}
%This work is licensed under the Creative Commons Attribution 3.0 Germany
%License. To view a copy of this license,
%visit \href{http://creativecommons.org/licenses/by/3.0/de/}{http://creativecommons.org} or send a letter
%to Creative Commons, 171 Second Street, Suite 300, San
%Francisco, California 94105, USA.

\end{document}
