\chapter{Introduction}
RoboCup is a competition for soccer robot held annually, which aims at promoting researches on robot and artificial intelligence. We work on the RoboCup Standard Platform League (SPL) by using humanoid robot NAO. There are 5 players for each side: goalkeeper, defender, supporter and striker. Different roles share some same functions such as self-localization, tracking the ball, kicking and so on. They also have different strategies according their distribution on the field just like the real soccer match. Of course the necessary communication and coorperation are required to achieve better performance in the match. One of the most successful team is B-Human from University of Bremen. They have won the championship five times. Last year TU Munich also set up our own RoboCup team called ``TUM Lion".
\section{Motivation}
The course ``Introduction Lab Humanoid RoboCup" in this semester is given for the students are interested in RoboCup and gives them a brief idea about RoboCup and the general knowledge of practice on autonomous robot loacalization, computer vision, motion control and multi-robot cooperation tasks. This semeater's project is based on the code framework from University of Bremen's team B-Human. With the help of the open source framework could let us start the project more quickly and do some specific reasearch on the topic what we are interested in or more familiar with. At the same time, some issues will be found and could be improved or considered as another new strategy.

\section{My Work}
Since we have already get the access to the state-of-the-art open source framework, so the first stage is to get familiar with the procedure of NAO operating and the codes. It includes:
\begin{itemize}
   	\item Basic operation on NAO and remote control through SimRobot
   	\item Calibration
	\item Understand how the current framework realize the self-localization
\end{itemize}

The second stage is focusing on self-localization and finding some interesting part to improve. Firstly I learned the Markov localization theory of mobile robot such as particle filter and unscented Kalman filter, which is used in the B-Human framework. Then the PnP pose estimation method is adopted to elimanate the current visual odometry error. In Chapter 2 I will introduce how does the current method figure out the localization of robot and the procedure of particle filer and unscented Kalman filter. In Chapter 3 I will show the visual odometry procedure of B-Human and explain what kind of noise it could be as well as the method that I applied.
