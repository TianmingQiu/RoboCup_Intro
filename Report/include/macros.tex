% using the todonotes package in some nice ways (see packages.tex)
\newcommand{\add}[2][]{\todo[color=blue!40,#1]{#2}{}}
\newcommand\optional[2][]{\todo[inline, color=cyan!40, caption={2do},
	#1]{\begin{minipage}{\textwidth-4pt}#2\end{minipage}}{}}

% math macros
\newcommand{\set}[1]{\boldsymbol{#1}}
\renewcommand{\vec}[1]{\mathbf{#1}}

% % commands for easy referencing
\newcommand{\fref}[1]{Figure~\ref{#1}}
\newcommand{\tref}[1]{Table~\ref{#1}}
\newcommand{\eref}[1]{Equation~\ref{#1}}
\newcommand{\cref}[1]{Chapter~\ref{#1}}
\newcommand{\sref}[1]{Section~\ref{#1}}
\newcommand{\aref}[1]{Appendix~\ref{#1}}

% fancy source code listings: http://stackoverflow.com/questions/741985/latex-source-code-listing-like-in-professional-books
% usage: \lstinputlisting[label=samplecode,caption=A sample]{sourceCode/HelloWorld.java}
\definecolor{light-gray}{gray}{0.95}
\usepackage{listings}
\usepackage{courier}
\lstset{
         basicstyle=\footnotesize\ttfamily, % Standardschrift
         numbers=left,               % Ort der Zeilennummern
         numberstyle=\tiny,          % Stil der Zeilennummern
         stepnumber=0,               % Abstand zwischen den Zeilennummern
         numbersep=5pt,              % Abstand der Nummern zum Text
         tabsize=2,                  % Groesse von Tabs
         extendedchars=true,         %
         breaklines=true,            % Zeilen werden Umgebrochen
         keywordstyle=\color{red},
         stringstyle=\color{white}\ttfamily, % Farbe der String
         showspaces=false,           % Leerzeichen anzeigen ?
         showtabs=false,             % Tabs anzeigen ?
         xleftmargin=0pt,
         framexleftmargin=10pt,
         framexrightmargin=10pt,
         framexbottommargin=0pt,
         backgroundcolor=\color{light-gray},
         showstringspaces=false      % Leerzeichen in Strings anzeigen ?
}

\lstdefinestyle{customc}{
  	belowcaptionskip=1\baselineskip,
  	breaklines=true,
  	%frame=L,
  	language=C,
  	showstringspaces=false,
 	basicstyle=\footnotesize\ttfamily\color{blue!40!black},
 	keywordstyle=\bfseries\color{green!40!black},
  	commentstyle=\itshape\color{purple!40!black},
  	%identifierstyle=\color{blue}, %color of actual code
 	stringstyle=\color{orange},
    numbers=left,               % Ort der Zeilennummern
    numberstyle=\tiny,          % Stil der Zeilennummern
    stepnumber=2,               % Abstand zwischen den Zeilennummern
    numbersep=5pt,              % Abstand der Nummern zum Text
    tabsize=2,                  % Groesse von Tabs
    extendedchars=true,         %
    showspaces=false,           % Leerzeichen anzeigen ?
    showtabs=false,             % Tabs anzeigen ?
  	xleftmargin=\parindent,
    framexleftmargin=10pt,
    framexrightmargin=10pt,
    framexbottommargin=0pt,
    backgroundcolor=\color{light-gray},
}

\lstset{escapechar=@,style=customc}
